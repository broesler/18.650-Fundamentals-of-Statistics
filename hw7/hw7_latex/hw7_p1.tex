\documentclass[letterpaper, oneside, reqno]{amsart}
\usepackage[english]{babel}
\usepackage[T1]{fontenc}
\usepackage[comma, compress, numbers, square]{natbib}
\usepackage{graphicx}
\usepackage{verbatim}
\usepackage{amsmath}
\usepackage{amssymb}
\usepackage{amsthm}
\usepackage{bbm}  % for indicator function
\usepackage{mathtools}
\usepackage{mleftright}
\usepackage{enumerate}
\usepackage{array, multirow}
\usepackage{url}
\usepackage[font=small, hypcap=true]{caption}                % link to top of figures and subfigures
\usepackage[font=small, hypcap=true, list=true]{subcaption}   % use for subfigures instead of {subfigure}

\usepackage{hyperref}
\hypersetup{
    linktoc=all,    % link table of contents to sections
    colorlinks,
    allcolors=black }

% Declare theorem environments
\newtheorem{thm}{Theorem}[section]
\newtheorem{prop}{Proposition}[section]

% lists as 1., 2., ...
\renewcommand{\labelenumi}{\theenumi.}

% Letter subsections
\renewcommand{\thesubsubsection}{\thesubsection(\alph{subsubsection})}

%-------------------------------------------------------------------------------
%     {listings + color} package options
%-------------------------------------------------------------------------------
% Define colors for code
\usepackage{xcolor}
\definecolor{gray}{rgb}{0.5, 0.5, 0.5}
\definecolor{mauve}{rgb}{0.58, 0, 0.82}
\definecolor{lblue}{HTML}{1F77B4}

\usepackage{listings}
\lstset{
  language = Python,                      % the language of the code
  basicstyle = \scriptsize\ttfamily,    % the size of the fonts that are used for the code
  numbers = left,                         % where to put the line-numbers
  numberstyle = \tiny\color{gray},        % the style that is used for the line-numbers
  stepnumber = 1,                         % the step between two line-numbers.
  numbersep = 8pt,                        % how far the line-numbers are from the code
  breaklines = true,                      % sets automatic line breaking
  keywordstyle = \color{lblue},            % keyword style
  commentstyle = \color{gray},         % comment style
  stringstyle = \color{mauve}             % string literal style
}

% Place loose figures at actual top of page
% \makeatletter
%   \setlength{\@fptop}{0pt}
% \makeatother

%-------------------------------------------------------------------------------
% NO HYPHENATION
%-------------------------------------------------------------------------------
\tolerance=1
\emergencystretch=\maxdimen
\hyphenpenalty=10000
\hbadness=10000

%==============================================================================
% General macros
%==============================================================================
% \renewcommand{\L}{\left}
% \newcommand{\R}{\right}
\renewcommand{\(}{\mleft(}
\renewcommand{\)}{\mright)}
\renewcommand{\[}{\mleft[}
\renewcommand{\]}{\mright]}
\newcommand{\T}{\top} % transpose symbol

\newcommand{\code}[1]{\texttt{#1}}
\newcommand{\vect}[1]{\boldsymbol{\mathbf{#1}}} % use BOTH to cover all cases

% Misc.
\newcommand{\ie}{\emph{i.e.\ }}

% Use for fractions with large terms
\newcommand{\ddfrac}[2]{\frac{\displaystyle #1}{\displaystyle #2}}

% Derivatives
\newcommand{\dd}{\mathrm{d}}
\newcommand{\ddx}[2]{\frac{\dd #1}{\dd #2}}
\newcommand{\dx}[1]{\,\dd #1} % for inside integration

% Probability helpers
\newcommand{\Prob}[1]{\mathbb{P}\[#1\]}
\newcommand{\E}[1]{\mathbb{E}\[#1\]}
\newcommand{\V}[1]{\mathbb{V}\[#1\]}
\newcommand{\R}{\mathbb{R}}  % real numbers
\newcommand{\N}[2]{\mathcal{N}\( #1, #2 \)}
\newcommand{\U}[2]{\mathcal{U}\(\[ #1, #2 \]\)}
\newcommand{\indep}{\perp \!\!\! \perp}  % "is independent from"
\newcommand{\indic}[1]{\mathbbm{1}\!\mleft\{#1\mright\}} % indicator function
\newcommand{\iid}{i.i.d.\ }

\newcommand{\sumi}[2]{\sum_{#1=1}^{#2}}
\newcommand{\avg}[2]{\frac{1}{#2}\sumi{#1}{#2}}

\newcommand{\by}[1]{&\quad&\text{(#1)}}

\newcommand{\alim}{\xrightarrow[n \to \infty]{\text{a.s.}}}
\newcommand{\plim}{\xrightarrow[n \to \infty]{\mathbb{P}}}
\newcommand{\dlim}{\xrightarrow[n \to \infty]{(d)}}

%%%% TITLE ----------------------------
\title[Homework 7 -- Problem \thesection]{18.650 Fundamentals of Statistics\\{\large Homework 7}}
\author{Bernie Roesler}
\date{\today}
%%%%

%%%%%%% BEGIN DOCUMENT ----------------
\begin{document}
% \sloppy

\graphicspath{{./figures/}}

\maketitle

\section{QQ-Plots}
Consider the QQ-plots of five \iid random variables with the following
distributions:
\begin{enumerate}
\item Standard Normal, $\N{0}{1}$,
  \item Uniform distribution, $\U{-\sqrt{3}}{\sqrt{3}}$,
  \item Cauchy distribution $\sim g(x) = \frac{1}{\pi}\frac{2}{1+x^2}$,
  \item Exponential distribution $\sim \operatorname{Exp}(\lambda) = \lambda
    e^{-\lambda x}, \lambda = 1$,
  \item Laplace distribution $\sim \operatorname{Laplace}(\lambda)
    = \frac{\lambda}{2} e^{-\lambda x}, \lambda = \sqrt{2}$.
\end{enumerate}
Figure \ref{fig:qqplots} shows the samples labeled with the appropriate
distribution.

\begin{figure}[!h]
  \centering
  \includegraphics[width=\textwidth]{qqplots.pdf}
  \caption{QQ-plots of five \iid random variables from different distributions.}
  \label{fig:qqplots}
\end{figure}

\clearpage
\section{Kolmogorov-Smirnov Test for Two Samples}
Consider two independent samples $X_1, \dots, X_n$, and $Y_1, \dots, Y_m$ of
independent, real-valued, continuous random variables, and assume that the $X_i$'s
are \iid with some cdf $F$ and that the $Y_i$'s are \iid with some cdf $G$. Note that the
two samples may have different sizes (if $n \ne m$). We want to test whether $F = G$.
Consider the following hypotheses:
\begin{align*}
  H_0: ``F = G" \\
  H_1: ``F \ne G"
\end{align*}
For simplicity, we will assume that $F$ and $G$ are continuous and increasing.

\subsection{Example Experiment}
An example experiment in which testing if two samples are from the same
distribution is of interest may be encountered in a lab setting where we have
two devices for measurement, and wish to determine if the errors have the same
distribution for our analysis. 

\subsection{CDF Distributions}
Let 
\begin{align*}
  U_i &= F(X_i), \quad \forall i = 1, \dots, n, \\
  V_j &= G(Y_j), \quad \forall j = 1, \dots, n.
\end{align*}

\begin{prop}
  The distribution of the cdf of a continuous random variable is uniform on $[0,
  1]$.
\end{prop}

\begin{proof}
The distributions of $U_i$ and $V_j$ can be determined by finding their cdfs.
The cdf of $U_i$ is defined by $F_U(t) \coloneqq \Prob{U_i \le t}$. Assuming that $F(X)$ and $G(Y)$ are invertible, it follows that
\begin{alignat*}{3}
  \Prob{U_i \le t} &= \Prob{F(X_i) \le t} \by{definition of $U_i$} \\
                   &= \Prob{X_i \le F^{-1}(t)} \\
                   &= F(F^{-1}(t)) \by{definition of cdf} \\
                   &= t \\
  \therefore F_U(t) &= t \\
  \implies f_U(t) &= \U{0}{1} \qedhere
\end{alignat*}
\end{proof}
Likewise, $f_V(t) = \U{0}{1}$.

\subsection{Empirical CDFs}
Let $F_n$ be the empirical cdf of $\{X_1, \dots, X_n\}$ and $G_m$ be the
empirical cdf of $\{Y_1, \dots, Y_m\}$.

\subsubsection{}
Let
\begin{equation}
  T_{n,m} = \sup_{t \in \R} \left| F_n(t) - G_m(t) \right|
\end{equation}

\begin{prop}
  The test statistic $T_{n,m}$ can be written as the maximum value of a finite set of numbers.
\end{prop}

\begin{proof}
  By definition, the cdf
  \begin{alignat}{3}
    F(t) &= \Prob{X \le t} \quad \forall t \in \R \by{$X_i$'s are \iid\!\!} \\
           &= \E{\indic{X \le t}} \nonumber \\
    \avg{i}{n}(\cdot) & \dlim \E{\cdot} \by{LLN} \nonumber \\
    \implies F_n(t) &= \avg{i}{n} \indic{X_i \le t} \label{eq:F_n}
    \intertext{Likewise,}
    G_m(t) &= \avg{j}{m} \indic{Y_j \le t}. \label{eq:G_m}
  \end{alignat}
  \begin{equation}
    \therefore T_{n,m} &= \sup_{t \in \R} \left| \avg{i}{n} \indic{X_i \le t} - \avg{j}{m} \indic{Y_j \le t} \right|.
  \end{equation}
  The empirical cdfs~\eqref{eq:F_n}~and~\eqref{eq:G_m} can also be written
  \begin{alignat}{3}
    F_n(t) &= \#\{i=1, \dots, n \colon X_i \le t\} \cdot \frac{1}{n} \\
    G_m(t) &= \#\{i=1, \dots, m \colon Y_j \le t\} \cdot \frac{1}{m},
  \end{alignat}
  so the only values that the empirical cdfs can take are the discrete sets
  \begin{align}
    F_n(i) &= \frac{i}{n} \quad \forall i = 1, \dots, n \\
    G_m(j) &= \frac{j}{m} \quad \forall j = 1, \dots, m.
  \end{align}
  Therefore, the test statistic can be rewritten as the maximum value of
  a finite set of numbers:
  \begin{equation}
    \begin{split}
      T_{n,m} = \max_{i=0,\dots,n} \Bigg[
      &\max_{j=0,\dots,m} \left| \frac{i}{n} - \frac{j}{m} \right| \indic{Y_j \le X_i < Y_{j+1}}, \right. \\ 
      \left. &\max_{k=j+1, \dots, m} \left| \frac{i}{n} - \frac{k}{m} \right| \indic{Y_k \le X_{i+1}} \Bigg]
    \end{split}
  \end{equation}
  where the values $X_0 \coloneqq -\infty$ and $Y_0 \coloneqq -\infty$ are prepended to the otherwise finite realizations to simplify the algorithm.
\end{proof}

\clearpage
The following subroutine computes an array of test statistics $T_v(i)$ for each value of $X_i$. The test statistic $T_{n,m}$ is the maximum of these values.
\lstinputlisting[language=python, firstline=75, lastline=124]{../hw7_kstest.py}

\clearpage
An example two-sample KS-test is shown in Figure~\ref{fig:ks_test}.
\begin{figure}[!h]
  \centering
  \includegraphics[width=0.9\textwidth]{ks_test.pdf}
  \caption{The empirical cdfs of two independent random samples from $\N{0}{1}$ and $\N{0}{2}$.}
  \label{fig:ks_test}
\end{figure}

\subsubsection{}
\begin{prop}
  If $H_0$ is true, then
  $$ T_{n,m} = \sup_{0 \le x \le 1} \left| \avg{i}{n} \indic{U_i \le x}
- \avg{j}{m} \indic{V_j \le x} \right|.
\end{prop}

\begin{proof}
  By~\eqref{eq:F_n}~and~\eqref{eq:G_m},
  \begin{equation}
    T_{n,m} &= \sup_{t \in \R} \left| \avg{i}{n} \indic{X_i \le t} - \avg{j}{m} \indic{Y_j \le t} \right|.
  \end{equation}
  To show the proposition is true, we make a change of variable. Let
    $$ x = F(t). $$
  Then,
    $$ t \in \R \implies x \in [0, 1]. $$
    Since $F$ and $G$ are continuous and monotonically increasing by definition,
  \begin{alignat*}{3}
    X_i &\le t \\
    \iff F(X_i) &\le F(t) \\
      \iff U_i &\le x \by{definition}.
  \end{alignat*}
  Similarly,
  \begin{alignat*}{3}
    Y_i &\le t \\
    \iff G(Y_i) &\le G(t) \\
      F(t) &= G(t) \by{under $H_0$} \\
    \iff G(Y_i) &\le F(t) \\
      \iff V_i &\le x \by{definition}. \qedhere
  \end{alignat*}
\end{proof}

\subsubsection{}
\begin{prop}
  If $H_0$ is true, the joint distribution of $U_1, \dots, U_n, V_1, \dots, V_m$
  $(n+m)$ random variables is uniform on $[0, 1]$.
\end{prop}

\begin{proof}
  \begin{alignat*}{3}
    \Prob{U_i \le t} &= \Prob{F(X_i) \le t} \\
                     &= \Prob{F(X_1) \le t} \by{\iid\!\!} \\
                     &= \Prob{G(X_1) \le t} \by{under $H_0$} \\
                     &= \Prob{G(Y_1) \le t} \by{\iid\!\!} \\
                     &= F(F^{-1}(t)) \\
                     &= t \\
    \therefore F_U(t) &= F_V(t) = t \\
    \implies f_{U,V}(t) &= \U{0}{1} \qedhere
  \end{alignat*}
\end{proof}

% % INCLUDE CODE:
% \clearpage
% \subsection*{Code}
% \renewcommand{\baselinestretch}{1.0}
% \lstinputlisting[language=matlab]{../engs250_2_1_wallshear.m}

%===============================================================================
% References:
%===============================================================================
% \clearpage
% \lhead{References}
% \bibliographystyle{apalike}
% \bibliography{engg149_finalbib}

%===============================================================================
%   Source Code {{{
%===============================================================================
% \clearpage
% \appendix   % BEGIN APPENDIX NUMBERING
% \lhead{Roesler, Final Exam, Source Code} % clear "Problem #" header
%
% \renewcommand{\baselinestretch}{1.0}
%
% \section{Main Script for Problems 1--4} \label{app:code1}
% \ % keep '\ 'space here to include "Source Code" appendix header
% \lstinputlisting[language=matlab]{../engg149_final_mimo.m}
%
% }}}


\end{document}
%%%%%%%%%%%%%%%%%%%%%%%%%%%%%%%%%%%%%%%%%%%%%%%%%%%%%%%%%%%%%%%%%%%
