\documentclass[letterpaper, oneside]{amsart}
\usepackage[english]{babel}
\usepackage[T1]{fontenc}
\usepackage[comma, compress, numbers, square]{natbib}
\usepackage{graphicx}
\usepackage{verbatim}
\usepackage{amsmath}
\usepackage{amssymb}
% \usepackage{mathtools}
\usepackage{enumerate}
\usepackage{array, multirow}
\usepackage{url}
\usepackage[font=small, hypcap=true]{caption}                % link to top of figures and subfigures
\usepackage[font=small, hypcap=true, list=true]{subcaption}   % use for subfigures instead of {subfigure}

%-------------------------------------------------------------------------------
%     {hyperref} package options
%-------------------------------------------------------------------------------
\usepackage{hyperref}
\hypersetup{
    linktoc=all,    % link table of contents to sections
    colorlinks,
    allcolors=black }

%-------------------------------------------------------------------------------
%     {listings + color} package options
%-------------------------------------------------------------------------------
% Define colors for code
% \usepackage{color}
% \definecolor{dkgreen}{rgb}{0,0.6,0}
% \definecolor{gray}{rgb}{0.5,0.5,0.5}
% \definecolor{mauve}{rgb}{0.58,0,0.82}

\usepackage{listings}
% \lstset{
%   % language = Matlab,                      % the language of the code
%   basicstyle = \scriptsize\ttfamily,    % the size of the fonts that are used for the code
%   numbers = left,                         % where to put the line-numbers
%   numberstyle = \tiny\color{gray},        % the style that is used for the line-numbers
%   stepnumber = 1,                         % the step between two line-numbers.
%   numbersep = 10pt,                        % how far the line-numbers are from the code
%   breaklines = true,                      % sets automatic line breaking
%   showspaces = false,                     % show spaces adding particular underscores
%   showstringspaces = false,               % underline spaces within strings
%   showtabs = false,                       % show tabs within strings adding particular underscores
%   keywordstyle = \color{blue},            % keyword style
%   commentstyle = \color{dkgreen},         % comment style
%   stringstyle = \color{mauve}             % string literal style
% }

% Place loose figures at actual top of page
% \makeatletter
%   \setlength{\@fptop}{0pt}
% \makeatother

%-------------------------------------------------------------------------------
% NO HYPHENATION
%-------------------------------------------------------------------------------
\tolerance=1
\emergencystretch=\maxdimen
\hyphenpenalty=10000
\hbadness=10000

%==============================================================================
% General macros
%==============================================================================
% \renewcommand{\L}{\left}
% \newcommand{\R}{\right}
\renewcommand{\(}{\left(}
\renewcommand{\)}{\right)}
\renewcommand{\[}{\left[}
\renewcommand{\]}{\right]}
\newcommand{\T}{\top} % transpose symbol

\newcommand{\code}[1]{\texttt{#1}}
\newcommand{\vect}[1]{\boldsymbol{\mathbf{#1}}} % use BOTH to cover all cases

% Misc.
\newcommand{\ie}{\emph{i.e.\ }}

% Use for fractions with large terms
\newcommand{\ddfrac}[2]{\frac{\displaystyle #1}{\displaystyle #2}}

% Derivatives
\newcommand{\dd}{\mathrm{d}}
\newcommand{\ddx}[2]{\frac{\dd #1}{\dd #2}}
\newcommand{\dx}[1]{\,\dd #1} % for inside integration

% Probability helpers
\newcommand{\N}[2]{\mathcal{N}\left( #1, #2 \right)}
\newcommand{\U}[2]{\mathcal{U}\left(\left[ #1, #2 \right]\right)}
\newcommand{\indep}{\perp \!\!\! \perp}  % "is independent from"
\newcommand{\indic}[1]{\mathbbm{1}\!\left\{#1\right\}} % indicator function
\newcommand{\iid}{i.i.d.\ }


%%%% TITLE ----------------------------
\title[Homework 7 -- Problem \thesection]{18.650 Fundamentals of Statistics\\{\large Homework 7}}
\author{Bernie Roesler}
\date{\today}
%%%%

%%%%%%% BEGIN DOCUMENT ----------------
\begin{document}
% \sloppy

\graphicspath{{./figures/}}

\maketitle

\section{QQ-Plots}
Consider the QQ-plots of five \iid random variables with the following
distributions:
\begin{itemize}
\item Standard Normal, $\N{0}{1}$,
  \item Uniform distribution, $\U{-\sqrt{3}}{\sqrt{3}}$,
  \item Cauchy distribution $\sim g(x) = \frac{1}{\pi}\frac{2}{1+x^2}$,
  \item Exponential distribution $\sim \operatorname{Exp}(\lambda) = \lambda
    e^{-\lambda x}, \lambda = 1$,
  \item Laplace distribution $\sim \operatorname{Laplace}(\lambda)
    = \frac{\lambda}{2} e^{-\lambda x}, \lambda = \sqrt{2}$.
\end{itemize}
Figure \ref{fig:qqplots} shows the samples labeled with the appropriate
distribution.

% \begin{figure}[!h]
%   \centering
%   \includegraphics[width=0.75\textwidth]{myfilename.pdf}
%   \caption{My caption goes here.}
%   \label{fig:qqplots}
% \end{figure}

% % INCLUDE CODE:
% \clearpage
% \subsection*{Code}
% \renewcommand{\baselinestretch}{1.0}
% \lstinputlisting[language=matlab]{../engs250_2_1_wallshear.m}

%===============================================================================
% References:
%===============================================================================
% \clearpage
% \lhead{References}
% \bibliographystyle{apalike}
% \bibliography{engg149_finalbib}

%===============================================================================
%   Source Code {{{
%===============================================================================
% \clearpage
% \appendix   % BEGIN APPENDIX NUMBERING
% \lhead{Roesler, Final Exam, Source Code} % clear "Problem #" header
%
% \renewcommand{\baselinestretch}{1.0}
%
% \section{Main Script for Problems 1--4} \label{app:code1}
% \ % keep '\ 'space here to include "Source Code" appendix header
% \lstinputlisting[language=matlab]{../engg149_final_mimo.m}
%
% }}}


\end{document}
%%%%%%%%%%%%%%%%%%%%%%%%%%%%%%%%%%%%%%%%%%%%%%%%%%%%%%%%%%%%%%%%%%%
