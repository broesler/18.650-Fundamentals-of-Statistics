\documentclass[letterpaper, reqno]{amsart}
\usepackage[english]{babel}
\usepackage[T1]{fontenc}
\usepackage[comma, compress, numbers, square]{natbib}
\usepackage{graphicx}
\usepackage{verbatim}
\usepackage{amsmath}
\usepackage{amssymb}
\usepackage{amsthm}
\usepackage{bbm}  % for indicator function
\usepackage{centernot}
\usepackage{mathtools}
\usepackage{mleftright}
\usepackage{enumerate}
\usepackage{array, multirow}
\usepackage[section, ruled]{algorithm}
\usepackage{algpseudocode}
\usepackage{url}
\usepackage[font=small, hypcap=true]{caption}                % link to top of figures and subfigures
\usepackage[font=small, hypcap=true, list=true]{subcaption}   % use for subfigures instead of {subfigure}

\usepackage{hyperref}
\hypersetup{
    linktoc=all,    % link table of contents to sections
    colorlinks,
    allcolors=black,
    urlcolor=dblue,
  }

% Declare theorem environments
\newtheorem{theorem}{Theorem}[section]
\newtheorem{lemma}[theorem]{Lemma}
\newtheorem{prop}{Proposition}[section]

\renewcommand{\qedsymbol}{\ensuremath{\blacksquare}}

% new environment for proofs of claims within proofs
\newenvironment{subproof}[1][\proofname]{%
  \renewcommand{\qedsymbol}{\ensuremath{\square}}%
  \begin{proof}[#1]%
}{%
  \end{proof}%
}

\numberwithin{equation}{section}

% lists as 1., 2., ...
\renewcommand{\labelenumi}{\theenumi.}

% Letter subsections
\renewcommand{\thesubsubsection}{\thesubsection(\alph{subsubsection})}

% number only specific equation in, say, align* environment
\newcommand{\numberthis}{\addtocounter{equation}{1}\tag{\theequation}}

% Algorithm
% \renewcommand{\algorithmicrequire}{\textbf{Input:}}
% \renewcommand{\algorithmicensure}{\textbf{Output:}}
\algnewcommand\Assert{\State \textbf{assert} }

%-------------------------------------------------------------------------------
%     {listings + color} package options
%-------------------------------------------------------------------------------
% Define colors for code
\usepackage{xcolor}
\definecolor{gray}{rgb}{0.6, 0.6, 0.6}
\definecolor{mauve}{rgb}{0.58, 0, 0.82}
\definecolor{dblue}{HTML}{0645AD}
\definecolor{lblue}{HTML}{1F77B4}

\usepackage{listings}
\lstset{
  language = Python,
  basicstyle = \scriptsize\ttfamily,
  numbers = left,
  numberstyle = \tiny\color{gray},
  stepnumber = 1,
  numbersep = 8pt,
  breaklines = true,
  keywordstyle = \bfseries\color{lblue},
  commentstyle = \color{gray},
  % stringstyle = \color{mauve}
}

% Place loose figures at actual top of page
% \makeatletter
%   \setlength{\@fptop}{0pt}
% \makeatother

%-------------------------------------------------------------------------------
% NO HYPHENATION
%-------------------------------------------------------------------------------
\tolerance=1
\emergencystretch=\maxdimen
\hyphenpenalty=10000
\hbadness=10000

%==============================================================================
% General macros
%==============================================================================
\DeclarePairedDelimiter{\ceil}{\lceil}{\rceil}
\DeclarePairedDelimiter{\floor}{\lfloor}{\rfloor}

\newcommand{\T}{\top} % transpose symbol
\newcommand{\vect}[1]{\boldsymbol{\mathbf{#1}}} % use BOTH to cover all cases

% Misc.
\newcommand{\ie}{\emph{i.e.\ }}
\newcommand{\eg}{\emph{e.g.\ }}

% Use for fractions with large terms
\newcommand{\ddfrac}[2]{\frac{\displaystyle #1}{\displaystyle #2}}

% Derivatives
\newcommand{\dd}{\mathrm{d}}
\newcommand{\ddx}[2]{\frac{\dd #1}{\dd #2}}
\newcommand{\dx}[1]{\,\dd #1} % for inside integration

% Probability helpers
\newcommand{\Prob}[1]{\mathbb{P}\mleft[#1\mright]}
\newcommand{\E}[1]{\mathbb{E}\mleft[#1\mright]}
\newcommand{\V}[1]{\mathbb{V}\mleft[#1\mright]}
\newcommand{\Var}[1]{\operatorname{Var}\mleft(#1\mright)}
\newcommand{\Cov}[1]{\operatorname{Cov}\mleft(#1\mright)}
\newcommand{\R}{\mathbb{R}}  % real numbers
\newcommand{\N}[2]{\mathcal{N}\mleft( #1, #2 \mright)}
\newcommand{\U}[2]{\mathcal{U}\mleft(\mleft[ #1, #2 \mright]\mright)}
\newcommand{\indep}{\perp \!\!\! \perp}  % "is independent from"
\newcommand{\nindep}{\centernot\indep}
\newcommand{\indic}[1]{\mathbbm{1}\!\mleft\{#1\mright\}} % indicator function
\newcommand{\iid}{i.i.d.}

\newcommand{\sumi}[2]{\sum_{#1=1}^{#2}}
\newcommand{\avg}[2]{\frac{1}{#2}\sumi{#1}{#2}}

\newcommand{\by}[1]{&\quad&\text{(#1)}}

\newcommand{\Alim}{\xrightarrow[n \to \infty]{\text{a.s.}}}
\newcommand{\Plim}{\xrightarrow[n \to \infty]{\mathbb{P}}}
\newcommand{\Dlim}{\xrightarrow[n \to \infty]{(d)}}

\newcommand{\phat}{\hat{p}}
\newcommand{\qhat}{\hat{q}}
\newcommand{\rhat}{\hat{r}}

\newcommand{\Xnbar}{\overline{X}_n}
\newcommand{\Rnbar}{\overline{R}_n}
\newcommand{\Qnbar}{\overline{Q}_n}

\DeclareMathOperator{\Ber}{Ber}
% \DeclareMathOperator{\Var}{Var}
% \DeclareMathOperator{\Cov}{Cov}

%%%% TITLE ----------------------------
\title[Homework 7 -- Problem \thesection]{18.650 Fundamentals of Statistics\\{\large Homework 7}}
\author{Bernie Roesler}
\date{\today}
%%%%

%%%%%%% BEGIN DOCUMENT ----------------
\begin{document}
% \sloppy

\graphicspath{{./figures/}}

\maketitle

\section{QQ-Plots}
Consider the QQ-plots of five \iid\ random variables with the following
distributions:
\begin{enumerate}
\item Standard Normal, $\N{0}{1}$,
  \item Uniform distribution, $\U{-\sqrt{3}}{\sqrt{3}}$,
  \item Cauchy distribution $\sim g(x) = \frac{1}{\pi}\frac{2}{1+x^2}$,
  \item Exponential distribution $\sim \operatorname{Exp}(\lambda) = \lambda
    e^{-\lambda x}, \lambda = 1$,
  \item Laplace distribution $\sim \operatorname{Laplace}(\lambda)
    = \frac{\lambda}{2} e^{-\lambda x}, \lambda = \sqrt{2}$.
\end{enumerate}
Figure \ref{fig:qqplots} shows the samples labeled with the appropriate
distribution.

\begin{figure}[!h]
  \centering
  \includegraphics[width=\textwidth]{qqplots.pdf}
  \caption{QQ-plots of five \iid\ random variables from different distributions.}
  \label{fig:qqplots}
\end{figure}

\clearpage
\section{Kolmogorov-Smirnov Test for Two Samples}
Consider two independent samples $X_1, \dots, X_n$, and $Y_1, \dots, Y_m$ of
independent, real-valued, continuous random variables, and assume that the $X_i$'s
are \iid\ with some cdf $F$ and that the $Y_i$'s are \iid\ with some cdf $G$.
\footnote{Note that the two samples may have different sizes (if $n \ne m$).}
We want to test whether $F = G$.
Consider the following hypotheses:
\begin{align*}
  H_0 \colon ``F = G" \\
  H_1 \colon ``F \ne G"
\end{align*}
For simplicity, we will assume that $F$ and $G$ are continuous and increasing.

\subsection{Example Experiment}
An example experiment in which testing if two samples are from the same
distribution is of interest may be encountered in a lab setting where we have
two devices for measurement, and wish to determine if the errors have the same
distribution for our analysis.

\subsection{CDF Distributions}
Let
\begin{align*}
  U_i &= F(X_i), \quad \forall i = 1, \dots, n, \\
  V_j &= G(Y_j), \quad \forall j = 1, \dots, n.
\end{align*}

\begin{prop}
  The distribution of the cdf of a continuous random variable is uniform on $[0,
  1]$.
\end{prop}

\begin{proof}
The distributions of $U_i$ and $V_j$ can be determined by finding their cdfs.
The cdf of $U_i$ is defined by $F_U(t) \coloneqq \Prob{U_i \le t}$. Assuming that $F(X)$ and $G(Y)$ are invertible, it follows that
\begin{alignat*}{3}
  \Prob{U_i \le t} &= \Prob{F(X_i) \le t} \by{definition of $U_i$} \\
                   &= \Prob{X_i \le F^{-1}(t)} \\
                   &= F(F^{-1}(t)) \by{definition of cdf} \\
                   &= t \\
  \therefore F_U(t) &= t \\
  \implies f_U(t) &= \U{0}{1} \tag*{\qedhere}
\end{alignat*}
\end{proof}
Likewise, $f_V(t) = \U{0}{1}$.

\subsection{Empirical CDFs}
Let $F_n$ be the empirical cdf of $\{X_1, \dots, X_n\}$ and $G_m$ be the
empirical cdf of $\{Y_1, \dots, Y_m\}$.

\subsubsection{The Test Statistic}
Let
\[
  T_{n,m} = \sup_{t \in \R} \left| F_n(t) - G_m(t) \right|
\]

\begin{prop}
  The test statistic $T_{n,m}$ can be written as the maximum value of a finite set of numbers.
\end{prop}

\begin{proof}
  By definition, the cdf
  \begin{alignat*}{3}
    F(t) &= \Prob{X \le t} \quad \forall t \in \R \\
         &= \E{\indic{X \le t}}. \\
    \intertext{By the Law of Large Numbers, the expectation can be approximated
        by the sample average, so we can define the \emph{empirical cdf} as}
    F_n(t) &= \avg{i}{n} \indic{X_i \le t} \numberthis \label{eq:F_n}
    \intertext{Likewise,}
    G_m(t) &= \avg{j}{m} \indic{Y_j \le t}. \numberthis \label{eq:G_m}
  \end{alignat*}
  \[
    \therefore T_{n,m} = \sup_{t \in \R} \left| \avg{i}{n} \indic{X_i \le t} - \avg{j}{m} \indic{Y_j \le t} \right|.
  \]
  The empirical cdfs~\eqref{eq:F_n}~and~\eqref{eq:G_m} can also be written
  \begin{alignat}{3}
    F_n(t) &= \#\{i=1, \dots, n \colon X_i \le t\} \cdot \frac{1}{n} \\
    G_m(t) &= \#\{i=1, \dots, m \colon Y_j \le t\} \cdot \frac{1}{m},
  \end{alignat}
  so the only values that the empirical cdfs can take are the discrete sets
  \begin{align}
    F_n(i) &= \frac{i}{n} \quad \forall i = 1, \dots, n \\
    G_m(j) &= \frac{j}{m} \quad \forall j = 1, \dots, m.
  \end{align}
  Therefore, the test statistic can be rewritten as the maximum value of
  a finite set of numbers:
  \[
    \begin{split}
      T_{n,m} = \max_{i=0,\dots,n} \Bigg[
      &\max_{j=0,\dots,m} \left| \frac{i}{n} - \frac{j}{m} \right|
        \indic{Y^{(j)} \le X^{(i)} < Y^{(j+1)}}, \\
      &\max_{k=j+1, \dots, m} \left| \frac{i}{n} - \frac{k}{m} \right|
        \indic{Y^{(k)} \le X^{(i+1)}} \Bigg]
    \end{split}
  \]
  where $X^{(i)}$ is the $i^\text{th}$ value in the ordered set of data
  $X^{(1)} \le \cdots \le X^{(n)}$. The values $X^{(0)}, Y^{(0)} \coloneqq -\infty$
  are prepended to the otherwise finite realizations to simplify the
  computation.
\end{proof}

\clearpage
The following algorithm calculates the KS test statistic for two given samples.

\begin{algorithm}[H]
  \caption{Calculate the KS test statistic $T_{n,m}$ for two samples.}
  \label{alg:ks_stat}
  \begin{algorithmic}[1]
    \Require $X, Y$ are vectors of real numbers.
    \Ensure $0 \le T_{n,m} \le 1$.
    \Procedure{KS2Sample}{$X, Y$}
      \State $X_s \gets \{-\infty,$ \Call{Sort}{$X$}$\}$
      \State $Y_s \gets$ \Call{Sort}{$Y$}
      \State $n \gets \dim X_s$
      \State $m \gets \dim Y_s$
      \State $T_v \gets$ empty array of size $n$
      \ForAll{$i \in \{0, \dots, n\}$}
        \State $j \gets j$ + \Call{Rank}{$\{Y_s^{(\ell)}\}_{\ell=j}^m, X_s^{(i)}$} \Comment{Only search remaining $j$ values}
        \State $k \gets j$ + \Call{Rank}{$\{Y_s^{(\ell)}\}_{\ell=j}^m, X_s^{(\min(i+1, n))}$}
        \State $\displaystyle{T_v^{(i)} \gets
          \max\mleft(\left|\frac{i}{n} - \frac{j}{m}\right|,
                \left|\frac{i}{n} - \frac{k}{m}\right|\mright)}$
      \EndFor
      \State\Return $\max_i T_v$
    \EndProcedure
    \Function{Rank}{$A, k$}
      % \State {\bfseries assert} $A$ is sorted in ascending order.
      \Assert $A$ is sorted in ascending order.
      \State\Return $\#\{i=1,\dots,\dim A \colon k < A_i\}$
    \EndFunction
  \end{algorithmic}
\end{algorithm}

\clearpage
The following subroutine is an implementation of Algorithm~\ref{alg:ks_stat}. It
computes an array of values $T_v(i)$ for each value of $X_i$. The test statistic
$T_{n,m}$ is the maximum of these values.
\lstinputlisting[language=python,
  rangeprefix=\#\ <<,
  rangesuffix=>>,
  includerangemarker=false,
  linerange=begin__ks_2samp-end__ks_2samp
  ]{../hw7_kstest.py}

\clearpage
An example two-sample KS-test is shown in Figure~\ref{fig:ks_test}.
\begin{figure}[!h]
  \centering
  \includegraphics[width=0.9\textwidth]{ks_test.pdf}
  \caption{The empirical cdfs of two independent random samples from $\N{0}{1}$ and $\N{0}{2}$. The test statistic $T_{n,m}$ is shown by the double arrow.}
  \label{fig:ks_test}
\end{figure}

\subsubsection{The Null Hypothesis}
\begin{prop}
  If $H_0$ is true, then the test statistic
  \[ T_{n,m} = \sup_{0 \le x \le 1} \left| \avg{i}{n} \indic{U_i \le x}
- \avg{j}{m} \indic{V_j \le x} \right|, \]
  which is a function only of the cdfs.
\end{prop}

\begin{proof}
  By~\eqref{eq:F_n}~and~\eqref{eq:G_m},
  \begin{equation} \label{eq:Tnm_supt}
    T_{n,m} = \sup_{t \in \R} \left| \avg{i}{n} \indic{X_i \le t} - \avg{j}{m} \indic{Y_j \le t} \right|.
  \end{equation}
  To show the proposition is true, we make a change of variable. Let
    \[ x = F(t). \]
  Then,
    \[ t \in \R \implies x \in [0, 1]. \]
    Since $F$ and $G$ are continuous and monotonically increasing,
  \begin{alignat*}{3}
    X_i \le t &\iff F(X_i) \le F(t) \\
              &\iff U_i \le x \by{definition}.
  \end{alignat*}
  Similarly,
  \begin{alignat*}{3}
    Y_i \le t &\iff G(Y_i) \le G(t) \\
              &\iff G(Y_i) \le F(t) \by{under $H_0$} \\
              &\iff V_i \le x \by{definition}.
  \end{alignat*}
  Substitution of these expressions into~\eqref{eq:Tnm_supt} completes the
  proof.
\end{proof}

\subsubsection{The Joint Distribution of the Samples}
\begin{prop} \label{prop:Tnm}
  If $H_0$ is true, the joint distribution of $U_1, \dots, U_n, V_1, \dots, V_m$
  $(n+m)$ random variables is uniform on $[0, 1]$.
\end{prop}

\begin{proof}
  \begin{alignat*}{3}
    \Prob{U_i \le t} &= \Prob{F(X_i) \le t} \\
                     &= \Prob{F(X_1) \le t} \by{\iid} \\
                     &= \Prob{G(X_1) \le t} \by{under $H_0$} \\
                     &= \Prob{G(Y_1) \le t} \by{\iid} \\
                     &= \Prob{V_1 \le t} \by{definition} \\
    \intertext{These probabilities can be rearranged to find the cdfs of $U$ and $V$}
                     &= \Prob{X_1 \le F^{-1}(t)} \\
                     &= F(F^{-1}(t)) \by{definition of cdf} \\
                     &= t \\
    \therefore F_U(t) &= G_V(t) = t \\
    \implies f_{U,V}(t) &= \U{0}{1} \tag*{\qedhere}
  \end{alignat*}
\end{proof}

\subsubsection{The Test Statistic is Pivotal}
Since Proposition~\ref{prop:Tnm} has been shown to be true under the null hypothesis $H_0$, and the distributions of $U_i$
and $V_j$ have been shown to be $\U{0}{1}$ independent of the distributions of
the underlying samples $X_i$, $Y_j$, we conclude that $T_{n,m}$ is
\emph{pivotal}, \ie it does not itself depend on the unknown distributions of
the samples.

\clearpage
\subsubsection{Quantiles of the Test Statistic}
Let $\alpha \in (0, 1)$ and $q_\alpha$ be the $(1 - \alpha)$-quantile of the
distribution of $T_{n,m}$ under $H_0$. The quantile $q_\alpha$ is given by
\begin{align*}
  q_\alpha &= F^{-1}(1-\alpha) \\
           &= \inf\{x \colon F(x) \ge 1 - \alpha\} \\
           &\approx \min\{x \colon F_n(x) \ge 1 - \alpha\}, \quad n < \infty \\
           \implies q_\alpha \approx \hat{q}_\alpha &= \min_i \left\{
               T_{n,m}^{(i)} \colon \tfrac{i}{M} \ge 1 - \alpha \right\}
\end{align*}
where $M \in \mathbb{N}$ is large, and $T_{n,m}^{(i)}$ is the $i^\text{th}$
value in a sorted sample of $M$ test statistics. Thus, $q_\alpha$ can be
approximated by choosing $i = \ceil{M(1 - \alpha)}$. An algorithm to approximate
$q_\alpha$ given $\alpha$ is as follows.

\begin{algorithm}[H]
  \caption{Approximate $q_\alpha$, the $(1 - \alpha)$-quantile of the distribution of $T_{n,m}$ under $H_0$.}
  \label{alg:ks_q}
  \begin{algorithmic}
    \Require $n = \dim X$, $m = \dim Y$, $M \in \mathbb{N}$, and $\alpha \in (0, 1)$.
    \Ensure $q_\alpha \in [0, 1]$.
    \Procedure{KSQuantile}{$n, m, M, \alpha$}
      \State $T_v \gets$ empty array of size $n$
      \ForAll{$i \in \{0,\dots,M\}$}
        \State $X_s \gets$ sample of size $n$ from $\N{0}{1}$.
        \State $Y_s \gets$ sample of size $m$ from $\N{0}{1}$.
        \State $T_v^{(i)} \gets$ \Call{KS2Sample}{$X_s, Y_s$} \Comment{defined in Algorithm~\ref{alg:ks_stat}}
      \EndFor
      \State $T_{vs} \gets$ \Call{Sort}{$T_v$}
      \State $j \gets \ceil*{M(1 - \alpha)}$
      \State \Return $T_{vs}^{(j)}$
    \EndProcedure
  \end{algorithmic}
\end{algorithm}

A plot of the distribution of
\[ \frac{T_{n,m}^M - \overline{T}_{n,m}^M}{\sqrt{\Var{T_{n,m}^M}}} \]
is shown in Figure~\ref{fig:Tnm} in comparison to a standard normal. The test
statistic distribution is skewed to the left, and has a longer right tail than
the standard normal.
Since the asymptotic distribution of the test statistic is not readily found in
theory, we rely on simulation via Algorithm~\ref{alg:ks_q} to estimate the
quantiles.

\begin{figure}[!h]
  \centering
  \includegraphics[width=0.95\textwidth]{ks_dist.pdf}
  \caption{Empirical distribution of samples of the test statistic $T_{n,m}$.}
  \label{fig:Tnm}
\end{figure}

\subsubsection{The Hypothesis Test}
Given the aproximation for $\hat{q}_\alpha$ for $q_\alpha$ from
Algorithm~\ref{alg:ks_q}, we define a test with non-asymptotic level $\alpha$
for $H_0$ vs.\ $H_1$:
\[
  \delta_\alpha = \indic{T_{n,m} > \hat{q}_\alpha^{(n, M)}}
\]
where $T_{n,m}$ is found by Algorithm~\ref{alg:ks_stat}. The p-value for this
test is
\begin{align}
  \text{p-value} &\coloneqq \Prob{Z \ge T_{n,m}} \\
  &\approx \frac{\#\{j = 1, \dots, M \colon T_{n,m}^{(j)} \ge T_{n,m}\}}{M}
\end{align}
where $Z$ is a random variable distributed as $T_{n,m}$.

\clearpage
\section{Aside: Homework 6, Problem 3 -- Test of Independence for Bernoulli Random Variables}
Let $X, Y$ be two Bernoulli random variables, not necessarily independent, and
let $p = \Prob{X=1}$, $q = \Prob{Y=1}$, and $r = \Prob{X=1, Y=1}$.

\subsection{Condition for Independence}
\begin{prop}
  $X \indep Y \iff r = pq$.
\end{prop}

\begin{proof}
  Two random variables are independent iff
  \begin{equation}
    \Prob{X \cap Y} = \Prob{X}\Prob{Y} \label{eq:indep}
    % \iff \Prob{X} = \frac{\Prob{X,Y}}{\Prob{Y}} = \Prob{X | Y}
  \end{equation}
  By the given definition of the Bernoulli random variables,
  \begin{align*}
    \Prob{X \cap Y} &= \Prob{X, Y} = \Prob{X=1, Y=1} = r \\
    \Prob{X} &= \Prob{X=1} = p \\
    \Prob{Y} &= \Prob{Y=1} = q \\
    \therefore r = pq &\iff \Prob{X,Y} = \Prob{X}\Prob{Y} \\
    &\iff X \indep Y  \qedhere
  \end{align*}
\end{proof}

\subsection{Test for Independence}
Let $(X_1, Y_1), \dots, (X_n, Y_n)$ be a sample of $n$ \iid\ copies of $(X, Y)$
(\ie $X_i \indep X_j$ for $i \ne j$, but $X_i$ may not be independent of $Y_i$).
Based on this sample, we want to test whether $X \indep Y$, \ie whether $r = pq$.

\subsubsection{Estimators of $p, q, r$}
Define the estimators:
\begin{align*}
  \phat &= \avg{i}{n} X_i, \\
  \qhat &= \avg{i}{n} Y_i, \\
  \rhat &= \avg{i}{n} X_i Y_i.
\end{align*}

\begin{prop}
  These estimators $\phat$, $\qhat$, and $\rhat$ are consistent estimators of
  the true values $p$, $q$, and $r$.
\end{prop}

\begin{proof}
  To show that an estimator is \emph{consistent}, we must prove that it
  converges to the true value of the parameter in the limit as $n \to \infty$.
  Since the sequence of $X_i$'s are \iid, we can use the Weak Law of Large
  Numbers (LLN) to prove that $\phat$ converges to $p$.

  \begin{theorem}[Weak Law of Large Numbers] \label{eq:LLN}
    If the sequence of random variables $X_1, \dots, X_n$ are \iid, then
    \[ \avg{i}{n} X_i \Plim \E{X}. \]
  \end{theorem}

  The expectation of $X$ is given by
  \begin{alignat*}{3}
    \E{X} &= \E{\Ber(p)} \by{given} \\
          &= p \by{definition of Bernoulli r.v.} \\
    \therefore \avg{i}{n} X_i &\Plim p \by{LLN} \\
    \implies \phat &\Plim p.
  \end{alignat*}
  Likewise $\qhat \Plim q$.

  To show that $\rhat$ converges to $r$, let $R \coloneqq X Y$ be
  a Bernoulli random variable with parameter $r = \Prob{X=1, Y=1}$, so that the
  estimator
  \begin{equation} \label{eq:rhat}
    \rhat = \avg{i}{n} X_i Y_i = \avg{i}{n} R_i.
  \end{equation}
  Note that the values of $R_i$ \emph{are} \iid\ since each pair $(X_i, Y_i)$
  are \iid, even though $X_i$ and $Y_j$ may not be independent for $i \ne j$.
  As before, we apply the Law of Large Numbers to the average of $R_i$'s. The
  expectation of $R$ is
  \begin{alignat*}{3}
    \E{R} &= \E{\Ber(r)} \by{definition} \\
          &= r \by{definition of Bernoulli r.v.} \\
    \therefore \avg{i}{n} R_i &\Plim r \by{LLN} \\
    \implies \rhat &\Plim r.
  \end{alignat*}
  Thus, each estimator $(\phat, \qhat, \rhat)$ converges to its
  respective parameter $(p, q, r)$.
\end{proof}

\subsubsection{Asymptotic Normality of the Estimators}
\begin{prop} \label{prop:normality_pqr}
    The vector of estimators $(\phat, \qhat, \rhat)$ is asymptotically normal, \ie
    \[ \sqrt{n} ((\phat, \qhat, \rhat) - (p, q, r))
        \Dlim \N{0}{\Cov{(\phat, \qhat, \rhat)}}. \]
\end{prop}
\emph{Proof.} To prove that the vector of estimators is asymptotically normal, we employ the
Central Limit Theorem (CLT).

\begin{theorem}[Central Limit Theorem]
    Let $X_1, \dots, X_n$ be a sequence of \iid\ random vectors $X_i \in \R^k$,
    and $\Xnbar = \avg{i}{n} X_i$.
    Then,
    \[ \sqrt{n}(\Xnbar - \E{X}) \Dlim \N{0}{\Sigma}. \]
    where $\Sigma$ is the $k$-by-$k$ matrix $\Cov{X}$.
\end{theorem}

By the CLT,
\begin{equation} \label{eq:CLT}
  \sqrt{n} ((\phat, \qhat, \rhat) - \E{(\phat, \qhat, \rhat)}) \Dlim \N{0}{\Sigma}
\end{equation}
where $\Sigma$ is the 3-by-3 symmetric covariance matrix, defined as
\begin{equation}
  \Sigma \coloneqq
  \begin{bmatrix}
    \Var{\phat} & \Cov{\phat, \qhat} & \Cov{\phat, \rhat} \\
    \cdot & \Var{\qhat} & \Cov{\qhat, \rhat} \\
    \cdot & \cdot & \Var{\rhat}
  \end{bmatrix}.
\end{equation}

We first need to determine the expectations of the estimators.

\begin{prop}
  The expectation of the estimator $\phat$ is $\E{\phat} = p$.
\end{prop}
\begin{subproof}
  \begin{alignat*}{3}
    \E{\phat} &= \E{\avg{i}{n} X_i} \by{definition} \\
              &= \frac{1}{n} \sumi{i}{n} \E{X_i} \by{linearity of expectation} \\
              &= \frac{1}{n} n \E{X} \by{\iid} \\
              &= \E{\Ber(p)} \by{definition} \\
    \implies \E{\phat} &= p \by{definition} \tag*{\qedhere}
  \end{alignat*}
\end{subproof}
\noindent Similarly, $\E{\qhat} = q$ and $\E{\rhat} = r$. This proposition
also shows that the estimators are \emph{unbiased}, since $\E{\phat - p} = 0$,
\emph{etc}.

We now determine the entries in the covariance matrix to complete the proof of
asymptotic normality.
\begin{prop}
The variance of $\phat$ is given by $\Var{\phat} = \frac{1}{n} p(1 - p)$.
\end{prop}
\begin{subproof}
  Using the definition of $\phat$,
  \begin{alignat*}{3}
    \Var{\phat} &= \Var{\avg{i}{n} X_i} \by{definition} \\
                &= \frac{1}{n^2}\Var{\sumi{i}{n} X_i} \by{variance rule} \\
                &= \frac{1}{n^2}\sumi{i}{n}\Var{X_i} \by{\iid} \\
                &= \frac{1}{n^2}n\Var{X} \by{\iid} \\
   \therefore \Var{\phat} &= \frac{1}{n} p(1-p) \by{variance of $\Ber(p)$}
  \end{alignat*}
  Likewise, $\Var{\qhat} = \frac{1}{n} q(1 - q)$,
  and $\Var{\rhat} = \frac{1}{n} r(1 - r)$.
\end{subproof}

\begin{prop}
  The covariance of $\phat$ and $\qhat$ is given by $\Cov{\phat, \qhat} = r - pq$.
\end{prop}
\begin{subproof}
  \begin{alignat*}{3}
    \Cov{\phat, \qhat} &= \Cov{\avg{i}{n} X_i, \avg{i}{n} Y_i} \\
    &= \frac{1}{n^2} \Cov{\sumi{i}{n} X_i, \sumi{i}{n} Y_i} \by{covariance property} \\
    &= \frac{1}{n^2} \sumi{i}{n} \sumi{j}{n} \Cov{X_i, Y_j} \by{bilinearity of covariance} \\
    &= \frac{1}{n^2} n^2 \Cov{X, Y} \by{identically distributed} \\
    &= \Cov{X, Y}  \\
    &= \E{XY} - \E{X}\E{Y} \by{definition of covariance} \\
    &= \E{R} - \E{X}\E{Y} \by{definition of $R$} \\
    \therefore \Cov{\phat, \qhat} &= r - pq \tag*{\qedhere}
  \end{alignat*}
\end{subproof}

\begin{prop}
  The covariance of $\phat$ and $\rhat$ is given by $\Cov{\phat, \rhat} = r(1 - p)$.
\end{prop}
\begin{subproof}
  \begin{alignat*}{3}
    \Cov{\phat, \rhat} &= \Cov{\avg{i}{n} X_i, \avg{i}{n} R_i} \\
    &= \frac{1}{n^2} \Cov{\sumi{i}{n} X_i, \sumi{i}{n} R_i} \by{covariance property} \\
    &= \frac{1}{n^2} \sumi{i}{n} \sumi{j}{n} \Cov{X_i, R_j} \by{bilinearity of covariance} \\
    &= \frac{1}{n^2} n^2 \Cov{X, R} \by{identically distributed} \\
    &= \Cov{X, R}  \\
    &= \E{X R} - \E{X}\E{R} \by{definition of covariance} \\
    &= \E{X R} - pr \by{given} \\
    &= \E{X (X Y)} - pr \by{definition of $R$} \\
    \intertext{Since $X \sim \Ber(p) \in \{0, 1\}$, $X^2 = X$, so we have}
    &= \E{X Y} - pr \\
    &= r - pr \\
    \therefore \Cov{\phat, \rhat} &= r(1 - p) \tag*{\qedhere}
  \end{alignat*}
\end{subproof}
\noindent Similarly, $\Cov{\qhat, \rhat} = r(1 - q)$.

The entire asymptotic covariance matrix is then
\begin{equation} \label{eq:sigma}
  \Sigma =
  \begin{bmatrix}
    p(1-p) & r - pq & r(1-p) \\
    \cdot & q(1-q) & r(1-q) \\
    \cdot & \cdot & r(1-r)
  \end{bmatrix}.
\end{equation}

Since we have determined the expectation $\E{(\phat, \qhat, \rhat)} = (p, q,
r)$, and the covariance matrix $\Sigma$ in terms of $p$, $q$, and $r$, we
conclude that Proposition~\ref{prop:normality_pqr} is true, and the vector of
estimators $(\phat, \qhat, \rhat)$ is asymptotically normal. \qed

\subsubsection{The Delta Method}
\begin{prop} \label{prop:delta}
  \[ \sqrt{n}\mleft((\rhat - \phat\qhat) - (r - pq)\mright) \Dlim \N{0}{V} \]
  where $V$ depends only on $p$, $q$, and $r$.
\end{prop}

\begin{proof}
  Let $\hat{\theta}$ and $\theta$ be vectors in $\R^3$
  \[
    \hat{\theta} = \begin{bmatrix} \phat \\ \qhat \\ \rhat \end{bmatrix} \text{, and }
    \theta = \begin{bmatrix} p \\ q \\ r \end{bmatrix}.
  \]
  From our proof of Proposition~\ref{prop:normality_pqr}, we have
  \begin{alignat*}{3}
    \sqrt{n}(\hat{\theta} - \theta) &\Dlim \N{0}{\Sigma} \by{CLT} \\
    \implies \sqrt{n}(g(\hat{\theta}) - g(\theta)) &\Dlim \N{0}{\nabla g(\theta)^\T \Sigma
      \nabla g(\theta)} \by{Delta method}
  \end{alignat*}
  for any differentiable function $g \colon \R^k \to \R$, and $\Sigma$ given by
  Equation~\eqref{eq:sigma}.
  Define the function
  \begin{equation} \label{eq:g}
    g(u, v, w) = w - uv
  \end{equation}
  such that
  \begin{align*}
    g(\hat{\theta}) &= \rhat - \phat\qhat, \\
    g(\theta) &= r - pq.
  \end{align*}
  The gradient of $g(\theta)$ is then
  \[
    \nabla g(u,v,w) = \begin{bmatrix} -v \\ -u \\ 1 \end{bmatrix}
    \implies \nabla g(\theta) = \begin{bmatrix} -q \\ -p \\ 1 \end{bmatrix}
  \]
  The asymptotic variance $V = \nabla g(\theta)^\T \Sigma \nabla g(\theta)$,
  which we will now show is a function only of the parameters $(p, q, r)$.
  \begin{alignat*}{3}
    V &= \begin{bmatrix} -q & -p & 1 \end{bmatrix}
    \begin{bmatrix}
      p(1-p) & r - pq & r(1-p) \\
      \cdot & q(1-q) & r(1-q) \\
      \cdot & \cdot & r(1-r)
    \end{bmatrix}
    \begin{bmatrix} -q \\ -p \\ 1 \end{bmatrix} \numberthis \label{eq:V_mat} \\
    &= \begin{bmatrix} -q & -p & 1 \end{bmatrix}
    \begin{bmatrix}
      -qp(1-p) - p(r - pq) + r(1-p) \\
      -q(r - pq) - pq(1-q) + r(1-q) \\
      -qr(1-p) - pr(1-q) + r(1-r)
    \end{bmatrix} \\
    &= \begin{bmatrix} -q & -p & 1 \end{bmatrix}
    \begin{bmatrix}
      (r - pq)(1 - 2p) \\
      (r - pq)(1 - 2q) \\
      r((1-p)(1-q) - (r-pq))
    \end{bmatrix} \\
    \begin{split}
      &= -q(r - pq)(1 - 2p) - p(r - pq)(1 - 2q)) \\
      &\,\quad + r((1-p)(1-q) - (r-pq))
    \end{split} \\
    \therefore V &= (r - pq)[-q(1 - 2p) - p(1 - 2q) - r] + r(1-p)(1-q)
    \numberthis \label{eq:V}
  \end{alignat*}
  which is a function only of $(p, q, r)$.
\end{proof}

\subsubsection{The Null Hypothesis}
Consider the hypotheses
\begin{align*}
  H_0 \colon X \indep Y \\
  H_1 \colon X \nindep Y
\end{align*}

\begin{prop} \label{prop:V_H0}
  If $H_0$ is true, then $V = pq(1-p)(1-q)$.
\end{prop}

\begin{proof}
  Under $H_0$, $r = pq$. Using the previous expression for $V$,
  Equation~\eqref{eq:V}, replace $r$ by $pq$ to find
  \[ V = (pq - pq)[-q(1 - 2p) - p(1 - 2q) - pq] + pq(1-p)(1-q). \]
  The first term is identically 0, so
  \[ V = pq(1-p)(1-q). \qedhere \]
\end{proof}

\begin{prop} \label{prop:Vhat}
Given
  \[ V = pq(1-p)(1-q), \]
a consistent estimator is given by
  \[ \hat{V} = \phat\qhat(1 - \phat)(1 - \qhat). \]
\end{prop}

\begin{proof}
  To prove that $\hat{V}$ converges to $V$, we employ the Continuous Mapping
  Theorem.
  \begin{theorem}[Continuous Mapping Theorem]
    Let $X \in \R^n$ be a vector of random variables, and $g \colon \R^n \to \R$
    be a continuous function. Let $X_n = X_1, X_2, \dots$ be a sequence of
    random vectors. If $X_n \Plim X$, then $g(X_n) \Plim g(X)$.
  \end{theorem}

  Since $\phat \Plim p$ and $\qhat \Plim q$,
  $\hat{V}(\phat, \qhat) \Plim V(p, q)$.
\end{proof}

\subsubsection{A Hypothesis Test}
\begin{prop}
  Given $\alpha \in (0, 1)$, we propose the test statistic
  \[
    T_n \coloneqq \frac{\sqrt{n}(\rhat - \phat\qhat)}{\sqrt{\hat{V}}} \Dlim \N{0}{1}
  \]
  where $\hat{V}$ is given by Proposition~\ref{prop:Vhat}, and $t_{n-1}$ is
  Student's $t$-distribution with $n-1$ degrees of freedom.
\end{prop}

\begin{proof}
Proposition~\ref{prop:delta} gives the distribution of $g(\theta)$ (given by
Equation~\eqref{eq:g}) under $H_0$.
Assume that $p, q \in (0, 1)$ s.t.\ $V > 0$.
\begin{alignat*}{3}
  \sqrt{n}\mleft((\rhat - \phat\qhat) - (r - pq)\mright) &\Dlim \N{0}{V}
  \by{Proposition~\ref{prop:delta}} \\
  \sqrt{n}(\rhat - \phat\qhat) &\Dlim \N{0}{V} \by{$r = pq$ under $H_0$} \\
  \sqrt{n}\frac{(\rhat - \phat\qhat)}{\sqrt{V}} &\Dlim \N{0}{1} \numberthis \label{eq:Tn_norm}
\end{alignat*}
The asymptotic variance $V$, however, is unknown, so we divide the estimator by
$\sqrt{\frac{\hat{V}}{V}}$ to get an expression that will evaluate to our
desired test statistic
\[
  T_n = \ddfrac{\sqrt{n}\frac{(\rhat - \phat\qhat)}{\sqrt{V}}}
               {\sqrt{\frac{\hat{V}}{V}}}
\]
Given this expression, we can determine the distribution of $T_n$.
Equation~\eqref{eq:Tn_norm} shows that the numerator is a standard normal random
variable.
\href{https://en.wikipedia.org/wiki/Cochran's_theorem}{\underline{Cochran's theorem}}
gives the distribution of the denominator.
\begin{lemma}[Result of Cochran's Theorem]
  If $X_1, \dots, X_n$ are \iid\ random variables drawn from the distribution
  $\N{\mu}{\sigma^2}$, and $S_n^2 \coloneqq \sumi{i}{n} (X_i - \Xnbar)^2$,
  then
  \[ \Xnbar \indep S_n, \]
  and
  \[ \frac{n S_n^2}{\sigma^2} \sim \chi^2_{n-1}. \]
\end{lemma}
Since $\hat{V}$ and $V$ describe the sample variance and variance of a
(asymptotically) normal distribution, $T_n$ is asymptotically characterized by
\[ T_n \Dlim \ddfrac{\N{0}{1}}{\sqrt{\frac{\chi^2_{n-1}}{n}}} \]
which is the definition of a random variable drawn from \emph{Student's
T-distribution} with $n-1$ degrees of freedom. In this case, however, the
normality of the underlying random variables is asymptotic, so the $t_{n-1}$
distribution approaches a standard normal distribution
\begin{align*}
  T_n &\Dlim t_{n-1} \\
  t_{n-1} &\Dlim \N{0}{1} \numberthis \label{eq:t_to_N} \\
  \implies T_n &\Dlim \N{0}{1} \qedhere
\end{align*}
A proof of Equation~\eqref{eq:t_to_N} is given in \nameref{app:t_to_N}.
\end{proof}

Given the test statistic $T_n$, define the rejection region
\[ R_\psi = \left\{ \hat{\theta} \colon |T_n| > q_{\alpha/2} \right\} \]
where
\[ q_{\alpha/2} = \Phi^{-1}\mleft(1 - \frac{\alpha}{2}\mright) \]
is the $\mleft(1-\frac{\alpha}{2}\mright)$-quantile of the standard normal $\N{0}{1}$
distribution.

We would like to know whether the facts of being happy and being in
a relationship are independent of each other. In a given population, 1000 people
(aged at least 21 years old) are sampled and asked two questions: ``Do you
consider yourself as happy?'' and ``Are you involved in a relationship?''. The
answers are summarized in Table~\ref{tab:tab1}.

\begin{table}[H]
  \setlength{\tabcolsep}{8pt}
  \def\arraystretch{1.1}
  \caption{}
  \label{tab:tab1}
    \centering
    \begin{tabular}{|r|c c|c|}
      \firsthline
                                  & {\bf Happy} & {\bf Not Happy} & {\bf Total} \\
      \hline
      {\bf In a Relationship}     & 205         & 301             & 506 \\
      {\bf Not in a Relationship} & 179         & 315             & 494 \\
      \hline
      {\bf Total}                 & 384         & 616             & 1000 \\
      \lasthline
    \end{tabular}
\end{table}

The values of our estimators are as follows:
\begin{align*}
  \phat &= \frac{\text{\# Happy}}{N} = \frac{384}{1000} = 0.384 \\
  \qhat &= \frac{\text{\# In a Relationship}}{N} = \frac{506}{1000} = 0.506 \\
  \rhat &= \frac{\text{\# Happy} \cap \text{\# In a Relationship}}{N} = \frac{205}{1000} = 0.205.
\end{align*}
The estimate of the asymptotic variance of the test statistic is
\[ \hat{V} = \phat\qhat(1-\phat)(1-\qhat) = (0.384)(0.506)(1 - 0.384)(1 - 0.506)
  = 0.05913, \]
giving the test statistic
\[ T_n = \frac{\sqrt{n}(\rhat - \phat\qhat)}{\sqrt{\hat{V}}}
  = \frac{\sqrt{1000}(0.205 - 0.384\cdot0.506)}{\sqrt{0.05913}} = 1.391. \]
The standard normal quantile at $\alpha = 0.05$ is $q_{\alpha/2}
= \Phi^{-1}\mleft(1
- \frac{\alpha}{2}\mright) = 1.96$, so the test result is
\[ |T_n| = 1.391 < q_{\alpha/2} = 1.96 \]
so we \emph{fail to reject $H_0$ at the 5\% level}. The $p$-value of the test is
\begin{align*}
  \text{$p$-value} &\coloneqq \Prob{Z > |T_n|} \by{$Z \sim \N{0}{1}$}  \\
                   &= \Prob{Z \le |T_n|} \by{symmetry} \\
                   &= \Prob{Z \le -T_n} + \Prob{Z > T_n} \\
                   &= 2\Phi(-|T_n|) \\
  \implies \text{$p$-value} &= 0.1642.
\end{align*}
In other words, the lowest level at which we could reject the null hypothesis is
at $\alpha = \text{$p$-value} = 0.1642 = 16.42\%$.

\clearpage
\subsection*{Appendix A: Additional Proofs} \label{app:t_to_N}
\begin{prop}
  A $t$-distribution with $n$ degrees of freedrom approaches a standard normal
  distribution as $n$ approaches infinity:
  \[ t_n \Dlim \N{0}{1}. \]
\end{prop}

\emph{Proof.}
Student's $t$-distribution with $\nu$ degrees of freedom is defined as the
distribution of the random variable $T$ such that
\[ t_{\nu} \sim T = \ddfrac{Z}{\sqrt{\frac{V}{\nu}}} \]
where $Z \sim \N{0}{1}$, $V \sim \chi^2_{\nu}$, and $Z \indep V$.

Let $X_1, \dots, X_n \sim \N{\mu}{\sigma^2}$ be a sequence of \iid\ random
variables. Define the sample mean and sample variance
\begin{align*}
  \Xnbar &\coloneqq \avg{i}{n} X_i \\
  S_n^2 &\coloneqq \avg{i}{n} (X_i - \Xnbar)^2.
\end{align*}
Let the random variables
\begin{align*}
  Z &= \frac{\sqrt{n}(\Xnbar - \mu)}{\sigma} \\
  V &= \frac{n S_n^2}{\sigma^2}.
\end{align*}
such that $Z \sim \N{0}{1}$ by the Central Limit Theorem, and $V \sim
\chi^2_{n-1}$ by Cochran's Theorem (which also shows that $Z \indep V$).
Then, the $t$-distribution is \emph{pivotal}
\[  t_{n-1} = \ddfrac{Z}{\sqrt{\frac{V}{n-1}}}. \]

\begin{lemma}
  The sample variance converges in probability to the variance,
  \[ S_n^2 \Plim \sigma^2. \]
\end{lemma}

\begin{subproof}
  \begin{alignat*}{3}
    S_n^2 &\coloneqq \avg{i}{n}(X_i - \Xnbar)^2 \\
          &= \avg{i}{n}(X_i^2 - 2 \Xnbar X_i + \Xnbar^2) \\
          &= \avg{i}{n} X_i^2 - \avg{i}{n} 2 \Xnbar X_i + \avg{i}{n} \Xnbar^2  \\
          &= \avg{i}{n} X_i^2 - 2 \Xnbar \avg{i}{n} X_i + \Xnbar^2  \\
          &= \avg{i}{n} X_i^2 - 2 \Xnbar^2  + \Xnbar^2  \\
          &= \avg{i}{n} X_i^2 - \Xnbar^2.
  \end{alignat*}
  The second term in the expression for $S_n^2$ is determined by
  \begin{alignat*}{3}
    \Xnbar &\Plim \E{X} \by{LLN} \\
    \E{X} &= \mu \by{given}. \\
    g(\Xnbar) &\Plim g(\mu) \by{CMT} \\
    \implies \Xnbar^2 &\Plim \mu^2.
  \end{alignat*}
  The first term in the expression for $S_n^2$ is then determined by
  \begin{alignat*}{3}
    \avg{i}{n} X_i^2 &\Plim \E{X^2} \by{LLN} \\
    \Var{X} &= \E{X^2} - \E{X}^2 \by{definition} \\
    \implies \E{X^2} &= \Var{X} + \E{X}^2 \\
                     &= \sigma^2 + \mu^2. \by{given} \\
    \therefore S_n^2 &\Plim \sigma^2 + \mu^2 - \mu^2 \\
    \implies S_n^2 &\Plim \sigma^2 \tag*{\qedhere}
  \end{alignat*}
\end{subproof}

Thus, $V \Plim \frac{n \sigma^2}{\sigma^2} = n$, a constant.

\begin{theorem}[Slutsky's Theorem]
  If the sequences of random variables $X_n~\Dlim~X$, and $Y_n~\Dlim~c$,
  a constant, then
  \begin{alignat*}{3}
    X_n + Y_n &\Dlim X + c \text{, and} \\
    X_n Y_n &\Dlim cX.
  \end{alignat*}
\end{theorem}

% TODO rewrite all in terms of unbiased sample variance
Since convergence in probability implies convergence in distribution, and $Z
\Dlim \N{0}{1}$, Slutsky's theorem implies that
\begin{align*}
  t_{n-1} = \ddfrac{Z}{\sqrt{\frac{V}{n-1}}}
        &\Dlim \ddfrac{\N{0}{1}}{\sqrt{\frac{n}{n-1}}} \\
        \implies t_{n-1} &\Dlim \N{0}{1}.  \qed
\end{align*}

\clearpage
\section{Test of Independence for Samples with Continuous CDF}
Consider the \iid\ pairs of random variables $(X_1, Y_1), \dots, (X_n, Y_n)$
with some continuous distribution. While each pair is independent, $X_i \indep
X_j$ for $i \ne j$, we would like to test if $X_i \indep Y_i$ for all $i$.

Define the hypotheses
\begin{align*}
  H_0 &\colon X_1 \indep Y_1 \\
  H_1 &\colon X_1 \nindep Y_1.
\end{align*}
For $i = 1, \dots, n$, let $R_i$ be the \emph{rank} of $X_i$ in the sample $X_1,
\dots, X_n$. The rank function is defined as
\[ R_i = \operatorname{rank}(X_i) \coloneqq \#\{j \colon X_j \le X_i\}  \]
\ie if $X_i = \min_j X_j$, then $R_i = 1$, and if $X_i = \max_j
X_j$, then $R_j = n$.
Similarly, let $Q_i$ be the rank of $Y_i$ in $Y_1, \dots, Y_n$.

\subsection{Example Experiment}
An example experiment in which testing for independence of two continuous random
variables is important is in a scientific experiment in which two devices are
measured using sensors powered by the same circuitry. We would like to ensure
that the measurements are not correlated.

\subsection{Dependence of the Ranks}
The ranks $R_1, \dots, R_n$ are \emph{not} independent because the rank of $X_i$
in the sample is unique. Therefore, if you know that $R_1 = 1$, \emph{e.g.}, then
$R_2,\dots,R_n \ne 1$.

\subsection{Proof of Distribution of Ranks} \label{subsec:3}
\begin{prop}
  The distribution of the vector $(R_1, \dots, R_n)$ does \emph{not} depend on
  the distribution of $X_i$'s (and, similarly, the distribution of $(Q_1,
  \dots, Q_n)$ does not depend on the distribution of $Y_i$'s).
\end{prop}

\begin{proof}
  Given the definition of $R_i$
  \[ R_i = \operatorname{rank}(X_i) \coloneqq \#\{j \colon X_j \le X_i\} \]
  we can determine its distribution.
  Let
  \[ F_n(t) = \avg{i}{n} \indic{X_i \le t} \]
  for some $t \in \R$ be the empirical cdf of the sample $X_i$'s.
  We can then rewrite the definition of the rank as
  \[ R_i = \sumi{j}{n} \indic{X_j \le X_i} = n F_n(X_i). \]
  The cdf of $R_i$, $F_R(t)$ is then
  \begin{align*}
    \Prob{R_i \le t} &= \Prob{n F_n(X_i) \le t} \\
                     &= \Prob{X_i \le F_n^{-1}\mleft(\tfrac{t}{n}\mright)} \\
                     &= F_n\mleft(F_n^{-1}\mleft(\tfrac{t}{n}\mright)\mright)
  \end{align*}
  Since $F_n$ is a piecewise-constant function, $F_n^{-1}$ does not exist. To
  avoid ambiguity, define
  \[ F_n^{-1}(p) = \inf\{x \colon F_n(x) \ge p\} \]
  where $p \in (0, 1)$. Given the definition of the empirical cdf, its inverse
  can only take the discrete values $X_1, \dots, X_n$, so
  \[  F_n\mleft(F_n^{-1}\mleft(\tfrac{t}{n}\mright)\mright)
      = \frac{\floor{t}}{n} \quad t \in [0, n] \]
  Therefore,
  \begin{align*}
    \Prob{R_i \le t} &= \frac{\floor{t}}{n} \quad t \in [0, n] \\
                    &= \mleft\{0, \tfrac{1}{n}, \tfrac{2}{n}, \dots, 1\mright\} \\
    \implies F_R(t) &= \mleft\{0, \tfrac{1}{n}, \tfrac{2}{n}, \dots, 1\mright\} \\
    \implies f_R(t) &= \mathcal{U}(\{0,\dots,n\}) \\
    \implies F_R(t) &\indep F_X(t) \qedhere
  \end{align*}
\end{proof}
An interpretation of this result is that $(R_1, \dots, R_n)$ is a permutation of
the vector $(1, \dots, n)$, with equal probability of any permutation.

\subsection{The Null Hypothesis} \label{subsec:4}
\begin{prop}
  If $H_0$ is true, then $(R_1, \dots, R_n) \indep (Q_1, \dots, Q_n)$.
\end{prop}

\begin{proof}
  Under $H_0$, $X_i \indep Y_i$, so
  \[ \Prob{X = x \land Y = y} = \Prob{X=x}\Prob{Y=y} \]
  Since the empirical cdfs $F_n$ and $G_n$ of $X_i$ and $Y_j$, respectively, are
  monotonically increasing, we can apply them to every value to get the expression
  \[ \begin{split}
      &\Prob{nF_n(X) = nF_n(x) \land nG_n(Y) = nG_n(y)} \\
      &\quad = \Prob{nF_n(X)=nF_n(x)}\Prob{nG_n(Y)=nG_n(y)}.
      \end{split}
  \]
  The rank $R_i = n F_n(X_i)$, and, similarly, $Q_j = n G_n(Y_j)$. The
  non-random values $nF_n(x)$ and $nG_n(y)$ are arbitrary integers in
  $(0, \dots, n)$, which we denote by $k$ and $\ell$. Therefore,
  \begin{align*}
    \Prob{R_i = k \land Q_j = \ell} &= \Prob{R_i = k}\Prob{Q_j = \ell} \\
    \implies R &\indep Q. \qedhere
  \end{align*}
\end{proof}

\subsection{Conclusion Under the Null Hypothesis} \label{subsec:5}
Under $H_0$, we have just shown that $R \indep Q$, so the joint distribution of
the $2n$ random variables $R_1, \dots, R_n, Q_1, \dots, Q_n$ is a discrete
uniform distribution on $[0, n]$, and does not depend on the distributions of
the $X$'s or $Y$'s.

\subsection{The Test Statistic} \label{subsec:6}
Consider the test statistic
\[ T_n \coloneqq \ddfrac{\sumi{i}{n}(R_i - \Rnbar)(Q_i - \Qnbar)}
                        {\sqrt{\sumi{i}{n} (R_i - \Rnbar)^2 \sumi{i}{n} (Q_i - \Qnbar)^2}}. \]
This test statistic is the empirical correlation between the two samples. If
$H_0$ is true, then $T_n \to 0$. We will now show that $T_n$ has a much simpler
expression.

\subsubsection{Simplifying the Rank Averages}
\begin{prop} \label{prop:rn1}
  \[ \Rnbar = \Qnbar = \frac{n+1}{2} \]
\end{prop}

\begin{proof}
  Because $R \indep Q$ and $R, Q$ are each permutations of $(1, \dots, n)$, the
  first equality is true.
  \begin{alignat*}{3}
    \Rnbar &\coloneqq \avg{i}{n} R_i \\
           &= \avg{i}{n} i \by{permutation of $(1, \dots, n)$} \\
           &= \avg{i}{n} Q_i \by{$R \indep Q$} \\
    \implies \Rnbar &= \Qnbar.
  \end{alignat*}
  Recognize the average as a geometric series
  \begin{alignat*}{3}
    \Rnbar &= \avg{i}{n} \\
           &= \frac{1}{n} \frac{n(n+1)}{2} \by{geometric series} \\
   \implies \Rnbar = \Qnbar &= \frac{n+1}{2}. \tag*{\qedhere}
  \end{alignat*}
\end{proof}

\begin{prop} \label{prop:rn2}
  \[  \sumi{i}{n} (R_i - \Rnbar)^2 = \sumi{i}{n} (Q_i - \Qnbar)^2 = \frac{n(n^2 - 1)}{12} \]
\end{prop}

\begin{proof}
  The first equality is true because $R, Q$ are each permutations of $(1, \dots, n)$.
  The sum evaluates to
  \begin{alignat*}{3}
    \sumi{i}{n} (R_i - \Rnbar)^2 &= \sumi{i}{n} (R_i^2 - 2\Rnbar R_i + \Rnbar^2) \\
    &= \sumi{i}{n} \mleft(i^2 - 2 i \frac{n+1}{2} + \mleft(\frac{n+1}{2}\mright)^2 \mright) \\
    &= \sumi{i}{n} i^2 - (n+1)\sumi{i}{n} i + \frac{(n+1)^2}{4} \sumi{i}{n} 1 \\
    \intertext{Using the fact that $\sumi{i}{n} i^2 = \frac{n(n+1)(2n+1)}{6}$,
      and the previous result,}
    &= \frac{n(n+1)(2n+1)}{6} - (n+1) \frac{n(n+1)}{2} + \frac{n(n+1)^2}{4} \\
    &= \frac{2n^3 + 3n^2 + n}{6} - \frac{n^3 + 2n^2 + n}{2} + \frac{n^3 + 2n^2 + n}{4}  \\
    &= \frac{1}{12} (4n^3 + 6n^2 + 2n - 6n^3 - 12n^2 - 6n + 3n^3 + 6n^2 + 3n) \\
    &= \frac{1}{12} (n^3 - n) \\
    &= \frac{n(n^2 - 1)}{12} \tag*{\qedhere}
  \end{alignat*}
\end{proof}

\subsubsection{The Test Statistic Simplified}
\begin{prop}
  The test statistic can be written as
  \[ T_n = \frac{12}{n(n^2 - 1)} \sumi{i}{n}R_iQ_i - \frac{3(n+1)}{n-1}. \]
\end{prop}

\begin{proof}
  Plugging the expressions from Propositions~\ref{prop:rn1} and~\ref{prop:rn2}
  into the definition of $T_n$,
  \begin{align*}
    T_n &\coloneqq \ddfrac{\sumi{i}{n}(R_i - \Rnbar)(Q_i - \Qnbar)}
                              {\sqrt{\sumi{i}{n} (R_i - \Rnbar)^2 \sumi{i}{n} (Q_i - \Qnbar)^2}} \\
    &= \ddfrac{\sumi{i}{n}\mleft(R_i - \frac{n+1}{2}\mright)\mleft(Q_i - \frac{n+1}{2}\mright)}
                              {\sqrt{\frac{n(n^2 - 1)}{12} \frac{n(n^2 - 1)}{12}}} \\
    &= \ddfrac{\sumi{i}{n}\mleft(R_iQ_i
                                - \frac{n+1}{2}(R_i + Q_i)
                                + \mleft(\frac{n+1}{2}\mright)^2 \mright)}
                              {\frac{n(n^2 - 1)}{12}} \\
    &= \frac{12}{n(n^2 - 1)} \mleft[ \sumi{i}{n}R_iQ_i
    + \mleft(\frac{n+1}{2}\mright)^2  \sumi{i}{n} 1
    - \frac{n+1}{2}\sumi{i}{n}(R_i + Q_i) \mright]
  \end{align*}
  The first term matches the first term of the proposition. The remaining two
  terms simplify to
  \begin{align*}
    &\phantom{=} \frac{12}{n(n^2 - 1)} \mleft[ \mleft(\frac{n+1}{2}\mright)^2  \sumi{i}{n} 1
    - \frac{n+1}{2}\sumi{i}{n}(R_i + Q_i) \mright]  \\
    &= \frac{12}{n(n^2 - 1)} \mleft[ \frac{n(n+1)^2}{4} - \frac{n+1}{2}\sumi{i}{n}2i \mright] \\
    &= \frac{12}{n(n^2 - 1)} \mleft[ \frac{n(n+1)^2}{4} - (n+1)\frac{n(n+1)}{2} \mright] \\
    &= \frac{12}{n(n+1)(n-1)} \mleft[ \frac{n(n+1)^2}{4} - \frac{n(n+1)^2}{2} \mright] \\
    &= \frac{12}{n-1} \mleft[ \frac{n+1}{4} - \frac{n+1}{2} \mright] \\
    &= \frac{12}{n-1} \mleft[ -\frac{n+1}{4} \mright] \\
    &= \frac{3(n+1)}{n-1} \\
    \implies T_n &= \frac{12}{n(n^2 - 1)} \sumi{i}{n}R_iQ_i - \frac{3(n+1)}{n-1}. \qedhere
  \end{align*}
\end{proof}

\subsection{The Distribution of the Test Statistic is Pivotal}
\begin{prop}
  If $H_0 \colon X \indep Y$ is true, then
  \[ T_n \sim S_n = \frac{12}{n(n^2 - 1)} \sumi{i}{n} R_i' Q_i' - \frac{3(n+1)}{n-1}, \]
  where $(R_1', \dots, R_n')$ and $(Q_1', \dots, Q_n')$ are ranks of two \iid\
  samples of $\U{0}{1}$.
\end{prop}

\begin{proof}
  Using the previous arguments, the proof is as follows:
  \begin{itemize}
    \item By the argument in \S\ref{subsec:3}, the distributions of $R_i', Q_i'$ do not depend on the distributions of the underlying random variables.
    \item By the argument in \S\ref{subsec:4}, the vector $(R_1', \dots, R_n')
      \indep (Q_1', \dots, Q_n')$ under $H_0$, so their distribution is known,
      and is the same discrete uniform distribution as $R, Q$.
    \item By the argument in \S\ref{subsec:6}, $T_n = S_n(R, Q)$.
  \end{itemize}
  Therefore, by the Continuous Mapping Theorem, $T_n \sim S_n$.
\end{proof}

\subsection{Computing Quantiles of the Test Statistic}
Let $\alpha \in (0, 1)$. Algorithm~\ref{alg:sn_q} approximates the $(1-\alpha)$-quantile
of $S_n$, $q_\alpha$.

\begin{algorithm}[H]
  \caption{Approximate $q_\alpha$, the $(1 - \alpha)$-quantile of the
distribution of $S_n$ under $H_0$.}
  \label{alg:sn_q}
  \begin{algorithmic}
    \Require $M, n \in \mathbb{N}$. $\alpha \in (0, 1)$.
    \Ensure $q_\alpha \in [0, 1]$.
    \Procedure{SnQuantile}{$n, M, \alpha$}
      \State $S_v \gets$ empty array of size $M$
      \ForAll{$i \in \{1,\dots,M\}$}
        \State $R \gets$ random permutation of $(1, \dots, n)$.
        \State $Q \gets$ random permutation of $(1, \dots, n)$.
        \State $S_v^{(i)} \gets \frac{12}{n(n^2 - 1)} \langle R, Q \rangle - \frac{3(n+1)}{n-1}$
      \EndFor
      \State $S_{vs} \gets$ \Call{Sort}{$S_v$}
      \State $j \gets \ceil*{M(1 - \alpha)}$
      \State \Return $S_{vs}^{(j)}$
    \EndProcedure
  \end{algorithmic}
\end{algorithm}

\subsection{A Non-Asymptotic Hypothesis Test}
Redefine the hypotheses in terms of the test statistic
\begin{align*}
  H_0 &\colon T_n = 0 \\
  H_1 &\colon T_n \ne 0
\end{align*}
with
\[ T_n = \frac{12}{n(n^2 - 1)} \sumi{i}{n}R_iQ_i - \frac{3(n+1)}{n-1}. \]
Let $\alpha \in (0, 1)$. The hypothesis test is then
\[ \delta_\alpha = \indic{|T_n| \ge \hat{q}_{\alpha/2}^{(n, M)}} \]
where $\hat{q}_{\alpha/2}^{(n, M)}$ is the estimated $(1 - \alpha)$-quantile
of $T_n$. The $p$-value for this test is
\begin{align*}
  \text{$p$-value} &\coloneqq \Prob{|Z| \ge |T_n|} \\
  &\approx \frac{\#\{j=1,\dots,M \colon |S_v^{(j)}| \ge |T_n|\}}{M}
\end{align*}
where $S_v^{(j)}$ is the $j^{\text{th}}$ sample of $S_n$, as computed in
Algorithm~\ref{alg:sn_q}.

A plot of the distribution of
\[ \frac{S_n^M - \overline{S}_n^M}{\sqrt{\Var{S_n^M}}} \]
is shown in Figure~\ref{fig:Sn} in comparison to a standard normal. The
underlying samples are each taken from an arbitrary
$\operatorname{Exp}(\lambda=1)$ to demonstrate the independence of the test
statistic distribution on the underlying sample distribution. This test
statistic is indeed normally distributed, although the parameters of the
distribution are not immediately apparent from our derivation. Since the
asymptotic distribution of the test statistic is not readily found in theory, we
rely on simulation via Algorithm~\ref{alg:sn_q} to estimate the quantiles.

\begin{figure}[!h]
  \centering
  \includegraphics[width=0.95\textwidth]{indep_dist.pdf}
  \caption{Plot of the distribution of the test statistic $S_n$, normalized to
    compare with a standard normal distribution.}
  \label{fig:Sn}
\end{figure}


% % INCLUDE CODE:
% \clearpage
% \subsection*{Code}
% \renewcommand{\baselinestretch}{1.0}
% \lstinputlisting[language=matlab]{../engs250_2_1_wallshear.m}

%===============================================================================
% References:
%===============================================================================
% \clearpage
% \lhead{References}
% \bibliographystyle{apalike}
% \bibliography{engg149_finalbib}

%===============================================================================
%   Source Code {{{
%===============================================================================
% \clearpage
% \appendix   % BEGIN APPENDIX NUMBERING
% \lhead{Roesler, Final Exam, Source Code} % clear "Problem #" header
%
% \renewcommand{\baselinestretch}{1.0}
%
% \section{Main Script for Problems 1--4} \label{app:code1}
% \ % keep '\ 'space here to include "Source Code" appendix header
% \lstinputlisting[language=matlab]{../engg149_final_mimo.m}
%
% }}}


\end{document}
%%%%%%%%%%%%%%%%%%%%%%%%%%%%%%%%%%%%%%%%%%%%%%%%%%%%%%%%%%%%%%%%%%%
